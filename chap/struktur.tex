\chapter{Struktur}

Die Arbeit kann mit den nachfolgenden Befehlen sehr einfach in Kapitel mit Unterkapiteln eingeteilt werden:
\begin{enumerate}
 \item \begin{verbatim}\chapter{Name}\end{verbatim} Beginne ein neues Kapitel
 \item \begin{verbatim}\section{Name}\end{verbatim} Beginne einen neuen Abschnitt eines Kapitels
 \item \begin{verbatim}\subsection{Name}\end{verbatim} Beginne einen Unterabschnitt eines Abschnitts in einem Kapitel
 \item \begin{verbatim}\subsubsection{Name}\end{verbatim} Beginne einen Unterabschnitt in einem Unterabschnitt
\end{enumerate}

Mithilfe von Labels lassen sich Referenzen erstellen: \begin{verbatim}\label{LabelName}\end{verbatim}
Um auf ein Label zu linken: \begin{verbatim}\ref{LabelName}\end{verbatim}

\section{Mein Abschnitt}
\subsection{Unterabschnitt}
\label{MeinUnterabschnitt}
In diesem Abschnit wird beschrieben\ldots

\subsection{Noch ein Unterabschnitt}
\subsubsection{Noch tiefere Hierarchie}
Im Unterabschnitt \ref{MeinUnterabschnitt} wurde erläutert\ldots
\subsubsection{Noch ein Unterpunkt vom Unterpunkt}
\subsection{Wieder eine Subsection}
\section{Und eine einfache Section}

\section{Zeilenumbrüche und Absätze}

Manchmal ist es ganz gut (hauptsächlich am Ende der Arbeit), die Strukturierung etwas in die Hand zu nehmen,
da LaTeX nicht immer eine optimale Aufteilung des Textes mit Grafiken und Tabellen erzeugt. 
Ein Zeilenumbruch lässt sich mit \begin{verbatim}\\\end{verbatim} erzeugen.\\
\\                                        
Falls ein Umbruch auf einen weiteren folgt, so erzeugt dies einen Absatz.                                      

\section{Neue Seite erzwingen}

Auch das erzwingen einer neuen Seite kann nützlich sein, um eine bessere Formatierung zu bekommen.
Mit \begin{verbatim}\newpage\end{verbatim} lässt sich dies erzwingen.
\newpage
Text auf neuer Seite.