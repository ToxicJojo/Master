\section{Page Rank}

PageRank bewertet Knoten in einem Graphen anhand von ihren Kanten. Knoten erhalten eine hohe Bewertung, wenn sie viele eingehenden Kanten besitzen. Auch die Bewertung der Knoten von denen die eingehenden Kanten ausgehen wirkt sich auf die Bewertung aus.


Da, die Daten verteilt auf mehren Servern liegen und keine globale Sicht existiert wird die iterative Variant von PageRank betrachtet.
Bei dieser verteilt jeder Knoten in jeder Iteration seinen eigenen PageRank Wert gleichmäßig auf alle Knoten zu denen er ausgehende Kanten besitzt. 


Sei $ PR(p_{i}; t)$ der PageRank Wert des Knoten $p_{i}$ zu dem Iterationsschrit $t$ und die gesamte Zahl aller Knoten $N$.
Zudem sei $M(p_{i})$ die Knoten die eine ausgehende Kante zu $p_{i}$ haben und $L(p_{i})$ die Anzahl an ausgehenden Kanten von Knoten $p_{i}$.
Jeder Knoten startet mit einem PageRank Wert von

\[  PR(p_{i}; 0) = \frac{1}{N}   \]

Bei jedem Iterationsschrit wird für jeden Knoten ein neuer Wert mit folgender Formel berechnet:

\[ PR(p_{i}; t+1) = \frac{1 - d}{N} + d \sum_{p_{j} \in M(p_{i})} \frac{PR(p_{j}; t)}{L(p_{j})} \]


Dabei ist $d$ ein Dämpfungsfaktor, der verhindert das einzelne Knoten, die keine ausgehenden Knoten besitzen die einzigen sind die keinen Wert von 0 am Ende haben.

Als Abbruchbedingung für die Iterationen können eine feste Anzahl an Iterationsschritten gewählt werden. Alternativ kann auch ein Grenzwert $\epsilon$ festgelegt werden und der Algorithmus bricht ab sobald die Veränderung aller Werte unter $\epsilon$ liegt.
