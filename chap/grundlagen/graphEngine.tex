\section{Graph Engine}

Graph Engine ist ein verteiltes in-Memory Datenverarbeitungssystem, welches von Microsoft Research Asia 
entwickelt wurde \cite{graphEngine}. Der Quellcode von Graph Engine ist quelloffen auf Github verfügbar und in C\# geschrieben.

Es bietet einen verteilten Key-Value Speicher, in dem Daten gespeichert und verarbeitet werden können. Dabei ist Graph Engine sehr flexibel darin, wie die Daten gespeichert werden. Der Anwender muss selbst
Schemata für die zu speichernden Daten erstellen. Dazu bietet Graph Engine die Möglichkeit, die Kommunikation unter den verschiedenen Servern mit selbst erstellten Protokollen zu koordinieren. Im Folgenden werden die verschiedenen Komponenten von Graph Engine besprochen.

\subsection{Memory Cloud}
\label{memoryCloud}

Im Kern von Graph Engine steht die sogenannte Memory Cloud. Diese stellt einen verteilten Key-Value Speicher dar, der im Arbeitsspeicher der Maschinen
liegt, um einen schnellen Zugriff zu gewährleisten.

Graph Engine verwaltet sogenannte Memory Trunks, in denen die Key-Value Paare gespeichert werden. Jeder der Trunks ist auf einer Maschine gespeichert.
In der Regel gibt es mehr Trunks als Maschinen, sodass eine Maschine mehrere Trunks hält. Durch die Aufteilung in mehrere Trunks kann parallel auf Daten aus verschiedenen Trunks zugegriffen
werden, ohne das ein weiterer Lock-Mechanismus benötigt wird. Die Größe der Trunks kann manuell gewählt werden, liegt aber bei den Standardeinstellungen bei 2GB.
Um zu gewährleisten, dass die Daten wiederhergestellt werden können, werden die Memory Trunks in dem verteilten Dateisystem Trinity File System (TFS) gespeichert. Das Design von TFS ist ähnlich zu Googles HDFS.

Als Schlüssel für die Werte werden 64-Bit Werte verwendet.
Die Werte selbst sind beliebig große Datenblobs, die direkt im Arbeitsspeicher der Maschinen verwaltet werden.


Um einen Wert anhand des Schlüssels zu finden, werden zwei Schritte durchgeführt. Zuerst wird die Maschine gefunden, die für den jeweiligen Schlüssel
verantwortlich ist. Danach wird der Schlüssel in den Memory Trunks dieser Maschine gefunden.

Im ersten Schritt wird der Schlüssel auf einen p-Bit Wert gehasht, um ein  $ i \in [0, 2^{p} - 1] $ zu erhalten. Der Schlüssel liegt demnach in
Memory Trunk $ i $. Jede Maschine besitzt eine Adressierungstabelle, die festhält, welcher Trunk auf welcher Maschine liegt.

Auf dieser Maschine muss nun der Schlüssel gefunden werden. Dafür besitzt jeder Memory Trunk eine Hashtabelle, die zu jedem Schlüssel
ein Offset und die Größe des Wertes im Speicher angibt.

\subsection{Graph Model}

Graph Engine bietet ein flexibles Model, mit dem die Graphdaten modelliert werden können. Es gibt keine festgelegte Struktur und es ist den Entwicklern
überlassen, Schemata für die Daten festzulegen. Hierbei hat man die Möglichkeit das Graphmodell genau an das zu lösende Problem anzupassen.
Dies bietet die Chance Optimierungen zu finden und gibt eine sehr feine Kontrolle über die gespeicherten Daten.

\subsubsection{Trinity Specification Language (TSL)}

Um das Datenmodell zu definieren, benutzt Graph Engine eine eigene Sprache, die Trinity Specification Language (TSL). Mit dieser werden sowohl die Schemas
für Daten, als auch Server Protokolle und Schnittstellen erstellt. 

TSL bietet die Möglichkeit Zellen zu definieren, welche im Betrieb als Werte im Key-Value Speicher abgelegt werden können.
Zellen können Grunddatentypen wie int, float, string etc. sowie Listen von Werten speichern. Um komplexere Werte darzustellen, können auch in TSL erstellte structs verwendet werden.
Mit diesen Möglichkeiten lassen sich sehr viele Datenstrukturen in TSL modellieren. 

Ein Beispiel für einen simplen Graphen ist in  \ref{lst:tsl} dargestellt. In diesem Beispielgraph hat jeder Knoten 
einen Wert und eine Liste an Kanten. Die Kanten selbst haben ein Gewicht sowie einen Verweis auf die ID des Knoten, auf den sie zeigen.

\begin{lstlisting}[language=c,label={lst:tsl}, caption={Beispiel für eine in TSL definierte Graphenstruktur}]
struct Edge {
  float Weight;
  long Link;
}

cell struct GraphNode {
  int Value;
  List<Edge> Edges;
}
\end{lstlisting}



Graph Engine hat einen eigenen Compiler für TSl, der die TSl Dateien in C\# Quellcode umwandelt. So werden aus den Definitionen Schnittstellen
generiert, um die entsprechenden Zellen in Graph Engine zu erstellen, zu verändern oder zu löschen. 

\subsection{Computation Engine}

Um Berechnungen durchzuführen, besteht ein Graph Engine Cluster aus drei verschiedenen Komponenten, die unterschiedliche Aufgaben übernehmen.


\subsubsection{Server}

Die Server in einem Graph Engine Cluster haben zwei Aufgaben: Zum einem speichern sie die Memory Trunks in ihrem Arbeitsspeicher, 
zum anderen führen sie Berechnungen auf den gespeicherten Daten durch.
Um diese Berechnungen durchzuführen, werden in der Regel Nachrichten mit anderen Graph Engine Komponenten ausgetauscht. Insbesondere die Kommunikation zwischen den Servern
selbst ist oft notwendig, da jeder Server nur eine Sicht auf seine lokal gespeicherten Daten hat.

\subsubsection{Proxy}

Proxies speichern selber keine Daten, können aber Nachrichten austauschen und Berechnungen durchführen. Sie können als 
Bindeglied zwischen Client und Server genutzt werden. So können sie z.B. von Clients geschickte Anfragen auf die Server aufteilen und deren
Berechnungen koordinieren oder die einzelne Ergebnisse aggregieren. Insbesondere in aufwändigeren Algorithmen kann eine Proxy 
den Ablauf kontrollieren und Entscheidungen wie Abbruchsbedingungen prüfen.

\subsubsection{Client}

Clients laufen auf der Maschine des Benutzers, der mit dem Graph Engine Cluster interagieren will. Clients senden Anfragen an die Server oder Proxies und
erhalten die entsprechenden Ergebnisse zurück. Sie halten keine Daten und führen in der Regel auch keine Berechnungen durch, womit es keine großen Hardwareanforderungen
an die Maschine gibt, auf der ein Client läuft.

\subsubsection{Protokolle}

Server, Proxies und Clients kommunizieren über Nachrichten, die sie einander schicken. Graph Engine unterstützt hierbei drei verschiedene Arten von Protokollen.


Synchrone Protokolle sind ähnlich zu synchronen Funktionaufrufen, die auf einer anderen Maschinen stattfinden. Sie blockieren die weitere Ausführung, bis
eine Antwort erhalten wurde. Ein Synchrones Protokoll kann sowohl in der Anfrage, als auch in der Antwort Daten mitsenden. So kann beispielsweise die Liste von
relevanten Schlüsseln übergeben werden und mit deren Gesamtsumme der Werte geantwortet werden.


Asynchrone Protokolle blockieren die Ausführung des Absenders nicht. Der Empfänger antwortet beim erhalten der Anfrage sofort mit der Bestätigung, dass diese erhalten wurde.
In einem Synchronen Protokoll können lediglich in der Anfrage Werte mitgegeben werden.
Der Empfänger startet einen Thread, der die Anfrage abarbeitet. Der Absender erfährt nicht, wann die Anfrage vollständig bearbeitet wurde.


HTTP Protokolle bieten Clients die Möglichkeit eine RESTful Version der Synchronen Protokolle über HTTP zu nutzen. Graph Engine erstellt automatisch die Endpunkte, an denen
auf Anfragen gewartet wird. So wird z.B. für ein Protokoll \verb|MyHTTProtocol| am Endpunkt

\verb|http://example.com/MyHttpProtocol| gewartet. Die Anfrage und
Antwort werden jeweils in JSON Strukturen übergeben. HTTP Protokolle werden nicht für Server zu Server Kommunikation genutzt, da sie deutlich weniger effizient sind als die
Synchronen und Asynchronen Protokolle.

\subsubsection{TSL}

Die Kommunikationschemata von Servern und Proxies werden in TSL definiert. In \ref{lst:ping} ist ein Beispiel für einen Server, der ein Synchrones Ping Protokoll unterstützt, dargestellt.

\begin{lstlisting}[language=c,label={lst:ping}, caption={In TSL definiertes Ping Protokoll}]
struct PingMessage {
  string Content;
}

protocol SynEchoPing {
  Type: Syn;
  Request: PingMessage;
  Response: PingMessage;
}


server PingServer {
  protocol SynEchoPing;
}
\end{lstlisting}

Wie schon bei den Zellen werden die Server und Protokolldefinitionen von dem TSL Compiler in C\# Code übersetzt. Für Server und Proxy
Definitionen werden abstrakte Klassen erstellt, in denen jeweils Methoden für die benötigten Protokolle implementiert werden müssen.
Es werden zudem Methoden generiert, um die definierten Anfragen an den Server oder die Proxy zu erstellen und diese zu senden.


\subsection{Datenzugriff}
\label{geAccessor}

Die Daten der Zellen liegen im Arbeitsspeicher der Maschinen als Datenblobs. Um auf diese komfortabel zuzugreifen, können die Daten in ein C\# Objekt
serialisiert werden; das ist jedoch sehr langsam.
Schnelleren Zugriff hat man, indem man die Daten direkt im RAM manipuliert. Das ist jedoch deutlich schwieriger, da man das Speicherlayout der jeweiligen Daten
kennen und entsprechende Zeiger-Arithmetik betreiben muss. Graph Engine löst diesen Konflikt, indem es aus der TSL Zellendefinition eine Zugriffklasse erzeugt.
Diese übersetzt die Lese- und Schreibzugriffe auf die Werte der Zelle auf die entsprechenden Operationen im Arbeitsspeicher. So lässt sich mit der Zugriffklasse sowohl
komfortabel, als auch effizient arbeiten.

\begin{lstlisting}[language=c, caption={Bearbeitung einer Zelle mithilfe der Zugriffklasse}]
  using (var node = Global.LocalStorage.UseNode(cellId, accessOptions)) {
    int value = node.Value;
    node.Value = 5;
  }
\end{lstlisting}
