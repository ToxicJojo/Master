\section{GraphEngine}

Graph Engine (GE) ist ein verteiltes in-Memory Datenverarbeitungssystem welches von Microsoft Research Asia 
entwickelt wurde. Es bietet einen verteilten Key-Value Speicher in dem Daten gespeichert und verarbeitet werden können

\subsection{Memory Cloud}

Im Kern von GE steht die sogenannte Memory Cloud. Diese stellt einen verteilten Key-Value Speicher dar, welcher im Arbeitsspeicher der Maschinen
liegt um einen schnellen Zugriff zu gewährleisten.
Als Schlüssel für die Werte werden 64 Bit Werte verwendet.
Die Werte sind beliebig große Datenblobs, welche direkt im Arbeitsspeicher der Maschinen verwaltet werden.

GE verwaltet sogenannte Memory Trunks, in denen die Key-Value Paare gespeichert werden. Jeder der Trunks ist auf einer Maschine gespeichert.
In der Regel gibt es mehr Trunks als Maschinen sodass eine Maschine mehrere Trunks hält. Durch die Aufteilung in mehrere Trunks kann parallel auf Daten aus verschiedenen Trunks zugegriffen
werden, ohne das ein weiterer Lock Mechanismus benötigt wird.
Um zu gewährleisten dass die Daten wiederhergestellt werden können, werden die Memory Trunks in dem verteilten Dateisystem Trinity File System (TFS) gespeichert. Das Design von TFS ist ähnlich zu Googles HDFS.



Um einen Wert anhand des Schlüssels zu finden werden zwei Schritte durchgeführt. Erst wird die Maschine gefunden, die für den jeweiligen Schlüssel
verantwortlich ist. Danach wird der Schlüssel in den Memory Trunks dieser Maschine gefunden.

Im ersten Schritt wird der Schlüssel auf einen p-Bit Wert gehasht um ein  $ i \in [0, 2^{p} - 1] $ zu erhalten. Der Schlüssel liegt demnach in
Memory Trunk $ i $. Jede Maschine besitzt eine Adressierungstabelle die fest hält welcher Trunk auf welcher Maschine liegt.

Auf dieser Maschine muss nun der Schlüssel gefunden werden. Dafür besitzt jeder Memory Trunk eine Hashtabelle die zu jeden Schlüssel
ein Offset und die Größe des Wertes im Speicher angbiet.

\subsection{Graph Model}

GE bietet ein flexibles Model mit denen die Graphdaten modelliert werden können. Es gibt keine festgelegte Struktur und es ist den Entwicklern
überlassen Schemata für die Daten festzulegen. Hierbei hat man die Möglichkeit das Graphmodell genau an das zu lösende Problem anzupassen.
Dies bietet die Chance Optimierungen zu finden und gibt eine sehr feine Kontrolle über die gespeicherten Daten.

\subsubsection{Trinity Specification Language (TSL)}

Um das Datenmodell zu definieren benutzt GE eine eigene Sprache, die Trinity Specification Language (TSL). Mit dieser werden sowohl die Schemas
für Daten als auch Server Protokolle und Schnittstellen erstellt. 

TSL bietet die Möglichkeit Zellen zu definieren, welche im Betrieb als Werte im Key-Value Speicher abgelegt werden können.
Zellen können Grunddatentypen wie int, float, string etc. speichern sowie Listen von Werten. Um komplexere Werte darzustellen können auch in TSL erstellte structs verwendet werden.
Mit diesen Möglichkeiten lassen sich sehr viele Datenstrukturen in TSL modellieren. Ein Beispiel für einen simplen Graphen ist in  \ref{lst:tsl} dargestellt.

\begin{lstlisting}[language=c,label={lst:tsl}, caption={Beispiel für einen in TSL definierte Graphenstruktur}]
struct Edge {
  float Weight;
  long Link;
}

cell struct GraphNode {
  int Value;
  List<Edge> Edges;
}
\end{lstlisting}



GE hat einen eigenen Compiler für TSl, der die TSl Dateien in C\# Quellcode umwandelt. So werden aus den Definitionen für Zellen Schnittstellen
generiert um diese in GE zu Erstellen, Verändern oder Löschen. 

\subsection{Computation Engine}

Um Berechnungen durchzuführen besteht ein GE Cluster aus drei verschiedenen Komponenten die unterschiedliche Aufgaben übernehmen.

\begin{enumerate}
  \item Server
  \item Proxy
  \item Client
\end{enumerate}


\subsubsection{Server}

Die Server in einem GE Cluster haben zwei Aufgaben. Zum einem speichern sie die Memory Trunks in ihrem Arbeitsspeicher, 
zum anderen führen sie Berechnungen auf den gespeicherten Daten durch.
Um diese Berechnungen durchzuführen werden in der Regel Nachrichten mit anderen GE Komponenten ausgetauscht. Insbesondere die Kommunikation zwischen den Servern
selbst ist oft notwendig, da jeder Server nur eine Sicht auf seine lokal gespeicherten Daten hat.

\subsubsection{Proxy}

Proxies speichern selber keine Daten können aber Nachrichten austauschen und Berechnungen durchzuführen. Sie können als 
Bindeglied zwischen Client und Server genutzt werden. So können sie z.B. von Clients geschickte Anfrangen auf die Server aufteilen und deren
Berechnungen koordinieren oder die einzelne Ergebnisse agreggieren. Insbesondere in aufwändigeren Algorithmen kann eine Proxy 
den Ablauf kontrollieren und Entscheidungen wie Abbruchsbedingungen prüfen.

\subsubsection{Client}

Clients laufen auf der Maschine des Benutzers der mit dem GE Cluster interagieren will. Clients senden Anfragen an die Server oder Proxies und
erhalten die entsprechenden Ergebnisse zurück. Sie halten keine Daten und führen in der Regel auch keine Berechnungen durch, womit es keine großen Hardwareanforderungen
an die Maschine gibt auf der ein Client läuft.

\subsubsection{Protokolle}

Server, Proxies und Clients kommunizieren über Nachrichten die sie einander schicken. GE unterstützt hierbei drei verschiedene Arten von Protokollen.


Syncrone Protokolle sind ähnlich zu syncronen Funktionaufrufen, die auf einer anderen Maschinen stattfinden. Sie blockieren die weitere Ausführung bis
eine Antwort erhalten wurde. Ein Synchrones Protokoll kann sowohl in der Anfrage als auch in der Antwort Daten mitsenden. So kann beispielsweise die Liste von
relevanten Schlüsseln übergeben werden und mit deren Gesamtsumme der Werte geantwortet werden.


Ansyncrhone Protokolle blockieren die Ausführung des Absenders nicht. Der Empfänger antwortet beim erhalten der Anfrage sofort mit der Bestätigung das diese erhalten wurde.
In einem Synchronen Protokoll können lediglich in der Anfrage Werte mitgegeben werden.
Der Empfänger startet einen Thread, der die Anfrage abarbeitet. Der Absender erfährt nicht wann die Anfrage vollständig bearbeitet wurde.


HTTP Protokolle bieten Clients die Möglichkeit eine RESTful Version der Syncronen Protokolle über HTTP zu nutzen. GE erstellt automatisch die Endpunkte an denen
auf Anfragen gewartet wird, so wird z.B. für ein Protokoll ``MyHTTProtocol'' am Endpunkt ``http://example.com/MyHttpProtocol'' gewartet. Die Anfrage und
Antwort werden jeweils in JSON Strukturen übergeben. HTTP Protokolle werden nicht für Server zu Server Kommunikation genutzt, da sie deutlich weniger effizient sind als die
Synrchonen und Asynchronen Protokolle.

\subsubsection{TSL}

Die Kommunikationschemas von Servern und Proxies werden in TSL definiert. Der folgende Block bietet ein Beispiel für einen Server der ein Synchrones Ping Protokoll unterstützt.

\begin{lstlisting}{C}
struct PingMessage {
  string Content;
}

protocol SynEchoPing {
  Type: Syn;
  Request: PingMessage;
  Response: PingMessage;
}


server PingServer {
  protocol SynEchoPing;
}
\end{lstlisting}

Wie schon bei den Zellen werden die Server und Protokolldefinitionen von dem TSL Compiler in C\# Code übersetzt. Für Server und Proxy
Definitionen werden abstrakte Klassen erstellt in denen jeweils Methoden für die benötigten Protokolle implementiert werden müssen.
Es werden zudem Methoden generiert um die definierten Anfragen an den Server oder die Proxy zu erstellen und diese zu senden.


\subsection{Datenzugriff}

Die Daten der Zellen liegen im Arbeitsspeicher der Maschinen als Datenblobs. Um auf diese komfortabel zuzugreifen können die Daten in ein C\# Objekt
serialisiert werden, das ist jedoch sehr langsam.
Schnelerren zugriff hat man indem man die Daten direkt im RAM manipuliert. Das ist jedoch deutlich schwieriger da man das Speicherlayout der jeweiligen Daten
kennen muss und entsprechende Zeiger Arithmetik betreiben muss. GE löst diesen Konflikt indem es aus der TSL Zellendefinition eine Zugriffklasse erzeugt.
Diese übersetzt die Lese und Schreibzugriffe auf die Werte der Zelle auf die entsprechenden Operationen im Arbeitsspeicher. So lässt sich mit der Zugriffklasse arbeiten sowohl
komfortabel als auch effizient arbeiten.

\begin{lstlisting}{language=sharpc}
  using (var node = Global.LocalStorage.UseNode(cellId, accessOptions)) {
    int value = node.Value;
    node.Value = 5;
  }
\end{lstlisting}
