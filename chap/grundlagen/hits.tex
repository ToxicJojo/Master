\section{Hubs and Authorities}

Hubs and Authorities (HITS) ist ein Algorithmus zur Bewertung von Knoten in einem Netzwerk, der von John M. Kleinberg in dem Paper ''Authoritative Sources a Hyperlinked Environment'' vorgestellt wurde \cite{Kleinberg98authoritativesources}.

Dabei werden in einem Netzwerk aufeinander verweisenden Doukumenten beurteilt. Beispiele für solche Netzwerke sind Internetseiten, die aufeinander
verlinken oder wissenschaftliche Veröffentlichung, die einander refenzieren. Ein Hub sind Dokumente, die auf viele gute Quellen verweisen, während
Authorities Doukumente gute Quellen sind, auf welche oft verwiesen wird. Der Algorithmus läuft ähnlich zum PageRank-Algorithmus ab, 
jedoch wird nicht nur ein einzelner Wert für jeden Knoten bestimmt, sondern jeweils ein Hub und Authority Werte pro Knoten.

Für einen gerichteten Graphen $G = (V, E)$ weisen wir jedem Knoten $v \in V$ einen Hub $y_{v}$ und Authority $x_{v}$ Wert zu.

HITS ist ein iterativer Algorithmus, der für jeden Knoten eines Graphen jeweils einen Hub und einen Authority Wert bestimmt. Hierfür 
werden die Hub oder Authority Werte der benachbarten Knoten genutzt. Damit die Authority und Hub Werte sich nicht gegen
unendlich aufschaukeln, werden sie in jeder Iterationen so normalisiert, dass die Summe der Quadrate 1 ergibt:

\[ \sum x_{i}^{2} = 1 \; \; \sum y_{i}^{2} = 1 \]

Die Werte für jeden Knoten konvergieren im Laufe der Iterationen.

Zu Beginn werden Hub und Authority Werte aller Knoten auf 1 gesetzt. Danach wird in jeder Iteration ein Authority und ein Hub Update durchgeführt.
\subsubsection{Autohrity Update}

Für jeden Knoten $ v \in V $ wird der neue Authority Wert aus der Summe der Hub Werte der eingehenden Kanten gebildet.

\[ x_{v} = \sum_{v; (u, v) \in E} y_{u} \]

Danach werden alle Authority Werte wie oben beschrieben normalisiert:

\[ x_{v} =  \frac{x_{v}}{\sum_{u \in V} x_{u}^{2}} \]

\subsubsection{Hub Update}

Für jeden Knoten $ v \in V $ wird der neue Hub Wert aus der Summe der Authority Werte der Knoten gebildet, auf die $v$ zeigt.

\[ y_{v} = \sum_{v; (v, u) \in E} x_{u} \]

Danach werden alle Hub Werte wie oben beschrieben normalisiert:

\[ y_{v} =  \frac{y_{v}}{\sum_{u \in V} y_{u}^{2}} \]

\subsubsection{Konvergenz}

Um den Algorithmus konvergieren zu lassen, wähle ein $ \epsilon $ und prüfe nach jeder Iteration, ob die gesammte Änderung der Authority
und Hub Werte kleiner als das $ \epsilon $ ist. Ist dies nicht der Fall, werden weitere Iterationen durchgeführt.


Alternativ kann auch eine feste Anzahl an Iterationen festgelegt werden. Sobald diese erreicht ist endet der Algorithmus.
