\section{HITS}

Hubs and Authorities (HITS) wurde von John M. Kleinberg in Authoritative Sources in a Hyperlinked Environment \footnote{\cite{Kleinberg98authoritativesources}} vorgestellt.

Dabei werden in einem Netzwerk aus aufeinander verweisenden Doukumenten beurteilt. Beispiele für solche Netzwerke sind Internetseiten die aufeinander
verlinken oder Wissenschaftliche Veröffentlichung die einander refenzieren. Ein Hub sind Dokumente die auf viele gute Quellen verweisen, während
Authorities Doukumente sind die gute Quellen sind und auf welche oft verwiesen wird. Der Algorithmus läuft ähnlich zum PageRank-Algorithmus ab 
jedoch wird nicht nur ein einzelner Werte für jeden Knoten bestimmt sonder jeweils ein Hub und Authority Werte pro Knoten. Umso höher der Wert umso
ein besser Hub oder Authority ist ein Knoten.

Für einen gerichteten Graphen $G = (V, E)$ weisen wir jedem Knoten $v \in V$ einen Authority $x_{v}$ und Hub $y_{v}$ Wert zu.

HITS ist ein iterativer Algorithmus der für jeden Knoten eines Graphen jeweils einen Hub und einen Authority Wert bestimmt. Hierfür 
werden jeweils die Hub oder Authority Werte der benachbarten Knoten genutzt. Damit die Authority und Hub Werte sich nicht gegen
unendlich aufschaukeln werden sie in jeder Iterationen so normalisiert das die Summe der Quadrate 1 ergibt:
$ \sum x_{i}^{2} = 1 $ und $ \sum y_{i}^{2} = 1$.
Die Werte für jeden Knoten konvergieren im laufe der Iterationen.

HITS aktualisiert zuerst die Authority Werte der Knoten und danach die Hub Werte.

Authority Update:

Für jeden Knoten $ v \in V $ wird der neue Authority Wert aus der Summe der Hub Werte der eingehenden Kanten gebildet.

\[ x_{p} = \sum_{q; (q, p) \in E} y_{q} \]


Danach werden alle Authority Werte wie oben beschrieben normalisiert:

$ x_{p} =  \frac{x_{p}}{\sum x_{q}^{2}} $

Hub Update:

Genauso


Um den Algorithmus konvergieren zu lassen wähle ein $ \epsilon $ und prüfe nach jeder Iteration ob die gesammte Änderung der Authority
und Hub Werte kleiner als das $ \epsilon $ ist. Ist dies nicht der Fall werden weitere Iterationen durchgeführt.
