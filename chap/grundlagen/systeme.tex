\section{Andere Multi Layer Graph Systeme}

Im folgendem werden verschiedenen andere Systeme betrachtet, mit denen Multilayer Graphen analysiert werden können.

\subsection{MuxViz}

Das Analysetool MuxViz wurde von De Domenico, M. und Porter, M. A. und Arenas, A. in ihrer Arbeit MuxViz: a tool for multilayer analysis and visualization of networks vorgestellt.
MuxViz ist ein open-source Projekt, welches es ermöglicht Multi Layer Graphen mit verschiedenen Algorithmen zu analysieren und visualisieren. Es nutzt für die Berechnungen R sowie GNU Octave und bietet mit einem modularen Aufbau die Möglichkeit, das Nutzer eigene Funktionalität hinzufügen.


MuxViz bietet eine Grafische Nutzeroberfläche, welche genutzt werden kann um Graphen zu laden, Algorithmen auzuführen und Graphen sowie Ergebnisse zu visualisieren. Die Benutzeroberfläche ist Webbasiert, was ermöglicht das die tatsächlichen Berechnung entweder lokal oder auf einem entfernten Server durchgeführt werden.
Dabei gibt es eine große Auswahl an Algorithmen und Statistiken die auf den Graphen angewandt werden können.


\begin{figure}
  \centering
  \begin{subfigure}[b]{1.0\textwidth}
    \includegraphics[width=1.0\linewidth]{img/muxViz.png}
  \end{subfigure}
  \caption{MuxVix}
  \label{muxVizSample}
\end{figure}

\subsection{Multilayer Networks Library for Python}

Die Multilayer Networks Library for Python (Pymnet) wurde von Mikko Kivelä 2015 veröffentlicht. 
Die in Python geschriebene Bibliothekt ermöglicht es mit Multilayer Netzwerke in Python zu arbeiten. Sie unterstützt das Laden und Manipulieren von Multilayer Netzwerken und biete eine Reihe an Algorithmen zur Analyse der Netzwerke.
Zudem können Netzwerke mithilfe von Matplotliv und D3 visualiert werden.
