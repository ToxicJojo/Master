
\section{Cluster}

Die Evaluation wurde auf dem Rechencluster des Lehrstuhl für Betriebssysteme der Heinrich-Heine Universität Düsseldorf durchgeführt.
Das Cluster besitzt mehrere Knoten die Nutzer reservieren und benutzen können.

Die zur Evalutation verwendeten Knoten haben den Xeon E3-1220 Prozessor, bezitzen 16Gigabyte Arbetsspeicher und eine 240 Gigabyte SSD.
Alle Knoten im Cluster sind mit Gigabit-Ethernet verbunden und können miteinander kommunizieren. 

Das home-Verzeichnis eines Nutzer ist in einem verteilten Dateisystem gespeichert. Damit können Dateien im home-Verzeichnis unabhänging davon welcher Knoten genutzt wird gelesen werden.

\section{Testdaten}


Als Testdatensatz wird ein Teil des Microsoft Acadmic Graph genutzt. Der Graph enthält wissenschaftliche Puplikation und die Zitierungsbeziehungen zwischen diesen Publikation, sowie Autoren, Institutionen, Journals, Konferenzen und Forschungsgebieten.

In dem Testdatensatz werden nur die Zitierungen zwischen Journals betrachtet. Dabei sind nur Journals enhalten in denen zwischen 2007 und 2011 mehr als 100 Paper veröffentlicht wurden und die mindestens auf andere Journals 5 mal verwiesen haben.
Der Datensatzt besteht aus einer Kantendatei, in die die Verweise von einem Journal auf ein anderes darstellen. Dabei speichert jede Kante welches Journals auf welches verweist, in welchem Forschungsgebiet, in welchem Jahr und die Anzahl der Verweise.

Die Kantendatei besteht aus X Kanten die zwischen Y Journals liegen.


\section{Ladezeiten}

