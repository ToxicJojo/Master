\chapter{Einleitung}

Nachfolgend einige Punkte, die es zu beachten gilt:
\begin{enumerate}
 \item Dieses Template dient als Vorlage, damit diverse Anforderungen wie allgemeines Layout der Arbeit,
       Formatierung etc. einheitlich sind. Weiterhin bietet es einen etwas leichteren Einstieg in LaTeX
       und sollte somit nicht zu viel Zeit in Anspruch nehmen, damit die eigentliche Arbeit nicht darunter leidet.
 \item Einzelne Kapitel dieser Vorlage geben einige grundlegende Konstrukte vor, die zum größten Teil per
       Copy/Paste übernommen werden können. Dennoch ist es unvermeidbar, sich mit LaTeX (bis zu einem
       gewissen Maß) zu beschäftigen. Mit Google kann man die meisten Probleme sehr leicht lösen und
       bekommt oft direkt den passenden Code geliefert.
\end{enumerate}

\subsection{Vor dem Druck/Finale Version}

Checkliste, was es vor dem Druck der finalen Version noch zu beachten gibt (keine Garantie auf Vollständigkeit!):
\begin{enumerate}
 \item Das Dokument ist auf doppelseitigen Druck ausgelegt, d.h. unbedingt doppelseitig drucken lassen.
 \item Vor dem Druck \textbf{alles} nochmal kontrollieren.
 \item Sind alle Kapitel vorhanden?
 \item Ist das Inhaltsverzeichnis vollständig?
 \item Sind alle Bestandteile der Arbeit vorhanden (Titelblatt, Inhaltsverzeichnis, Anhang, Erklärung, etc.)
 \item Sind leere Abschnitte entfernt (z.B. keine Algorithmen im Algorithmusverzeichnis)?
 \item Sind alle Todos aus dem Dokument entfernt und erledigt? 
 \item Ist das Todo Verzeichnis aus dem Dokument entfernt?
 \item Steht das richtige Datum auf der Arbeit?
 \item Das Thema muss \textbf{exakt} so formuliert sein, wie bei der Anmeldung der Arbeit.
 \item Die Arbeit ist gebunden in dreifacher Ausführung beim Prüfungsamt abzugeben.
 \item Jegliche digitalen Dokumente, Quellcode, Programme, Messergebnisse und Anleitungen für den Aufbau von Testumgebungen 
 sind jeder gedruckten Arbeit auf CD/DVD hinzuzufügen.  
\end{enumerate}